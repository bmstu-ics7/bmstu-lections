\section{Последовательность разработки и компьютерной реализации моделей системы}

Самостоятельно изучить команду дисперсионного анализа в GPRSS.

\textbf{Сущность компьютерного моделирования} состоит в проведении эксперимента с моделью, которая представляет собой прграммный комплекс, описывающий формально или алгоритмически поведение элементов системы в процессе функционирования системы (взаимодействие всех элементов друг с другом и внешней средой).

Самостоятельно блок-тест GPRSS (все возможные режимы).

\subsection{Основные требования представления модели}

\begin{enumerate}
    \item Полнота модели должна предоставлять пользователю возможность получения необходимого набора характеристик, оценок системы, с требуемой точностью и достоверностью.
    \item Гибкость модели должна давать возможность воспроизводить различные ситуации при варьированииструктуры, алгоритмов и параметров модели, причем структура должна быть блочной, то есть допускать возможность замены, добавления, искючения некоторых частей без переделывания всей системы.
    \item Компьютерная реализация модели должна соответствовать имеющимся техническим ресурсам (память, быстродействие, базы данных и т.д.)
\end{enumerate}

Процесс моделирования включает разработку и компьютерную реализацию модели является итерационным. Этот итерационный процесс продолжается до тех пор, пока не будет получена некоторая модель, которую можно считать адекватной в рамках решения поставленной задачи.

\subsection{Основные этапы моделирования больших систем}

\subsubsection{Построение концептуальной или описательной модели системы и ее формализация}

Формулируется модель и строится ее \textbf{формальная} схема. То есть основным содержанием этого этапа является переход от содержательного описания объекта к его математической модели. Данный этап является наименее формализованным. Исходный материал для данного этапа: содержательное описание объекта.

\paragraph{Последовательность действий}

\begin{enumerate}
    \item Проведение границы между системой и внешней средой
    \item Исследование моделируемого объекта с точки зрения выделения основных состовляющих процессов функционирования системы (по отношению к цели моделирования).
    \item Переход от содержательного описания системы к формализованному описанию свойств процессов функционирования, то есть собственно функциональной модели.
    \item Составленные модели системы группируются: блоки первой группы предстваляют собой имитатор воздействия внешней среды, блоки второй группы собственно модель функционирования системы, блоки третьей являются вспомогательными (интерпретация результатов моделирования)
    \item Процесс функционирования системы так разбивается на подпроцессы, чтобы построение отдельных процессов было элементарно и не вызывало особых трудностей (свели задачу к типовой математической схеме).
\end{enumerate}

\subsubsection{Алгоритмизация модели и компьютерная реализация}

На втором этапе математическая модель, сформулированная на первом этапе, перемещается в компьютер.

Исходный материал: блочная логическая схема модели.

\paragraph{Последовательность действий}

\begin{enumerate}
    \item Разработка схемы моделирующего алгоритма.
    \item Разработка схемы программы
    \item Выбор технических средств для реализации компьютерной модели.
    \item Программирования.
    \item Отладка.
    \item Проверка достоверности программы на тестовых примерах.
    \item Составление технической документации по второму этапу (логическая схема программы, текст, спецификация, затрачиваемые ресурсы).
\end{enumerate}

\subsubsection{Получение и интерпретация результатов}

Проводят рабочие расчеты по готовой программе, именно результаты этих расчетов позволят сделать вывод о характеристиках функционирования сложной системы.

\paragraph{Последовательность действий}

\begin{enumerate}
    \item Планирования компьютерного эксперимента с моделью (активный, пассивный). Составление плана проведения эксперимента с указанием комбинаций переменных, для которых проводится анализ функционирования. Главная задача здесь, дать максимальный объем информации при минимальных затратах вычислительных ресурсов.
    \item Проведение рабочих расчетов (контрольная калибровка модели).
    \item Обработка (статистическая) результатов, расчетов.
    \item Интерпретация результатов моделирования.
    \item Составление технической документации.
\end{enumerate}

Самостоятельно разобрать понятия тактического, стратегического прланирвоания эксперимента. Существенных факторов.

\subsubsection{Калибровка модели}

\begin{enumerate}
    \item Получить
    \item Сравнить с фактическим
    \item Смотрим допустима ошибка или нет (от параметрической до полного изменения структуры модели)
\end{enumerate}

\paragraph{3 класса ошибок при калибровке}

\begin{enumerate}
    \item Ошибки формализации (все заново)
    \item Ошибка решения (слишком упрощенный алгоритм, переходим к более точным методам)
    \item Ошибки параметров модели (варьируем до рабочей области)
\end{enumerate}

\paragraph{Проверка адекватности и корректировка модели}

Проверка адекватности заключается в анализе ее соразмерности и развозначности системы, Адекватность нарушается из-за идеализации внешних условий и режима функционирования, пренебрежение некоторыми случайными факторами.

Простейшая мера адекватности -- отклонение некоторой характеристики оригинала от модели. Считается что модель адекватна с ситемой, если вероятность того, чтоотклонение характеристики не превышает некоторой величины больше допустимой вероятности.

Фактическое использование одного критерия невозможно

\begin{enumerate}
    \item Нет информации о характеристике
    \item Система оценивается не по одной, а по множеству выходных характеристик.
    \item Характеристики могут быть случайными
    \item Очень сложно задать допустимое отклонение
\end{enumerate}

Выделяют следующие типы изменений

\begin{itemize}
    \item Глобальные -- в случае обнаружения методических ошибок
    \item Локальные -- уточнение некоторых параметров и алгоритмов
    \item Параметрические изменения
\end{itemize}

Завершается этот этап определением области пригодности модели, под которой понимается множество условий, при соблюдении которых точность результата находится в допустимых пределах.

\subsection{Схема взаимодействия технологических этапов моделирования}

В схеме две ошибки, нюансы
