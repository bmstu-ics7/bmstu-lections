\section{Философские основы моделирования}

\textbf{Методологическая основа моделирования} -- теоретический метод познания и начного исследования. Все то, на что направлена человеческая деятельность называется \textbf{объектом}. Научно-техническое развитие в любой области обычно идет по следующему пути:

\begin{enumerate}
    \item наблюдение и эксперимент;
    \item теоретическое исследование;
    \item организация производственых процессов.
\end{enumerate}

В научных исследованиях большую роль играют \textbf{гипотезы}, то есть опрделенные предсказания, основывающиеся на небольшом количестве опытных данных, наблюдениях, догадках. Быстрая и полная проверка выдвигаемых гипотез может быть проведена в ходе специально поставленного эксперимента. При формировании и проверке правильности гепотез большое значение в качестве метода суждения имеет \textbf{аналогия}, под кторой будем понимать суждения о каком-либо частном сходстве двух объектов. Современная научная гипотеза создается как правило по аналогии с проверенными на практике положениями. То есть аналогия связывают гипотезу с экспериментом.

Гипотезы и аналогии, отражающие реальный, объективно существующий мир, должны обладать \textbf{наглядностью} или сводиться к удобным для исследования лоигческим схемам. Такие логические схемы, упрощающие рассуждения и логические построения или позволяющие проводить эксперимент, уточняющий природу явлений, называются \textbf{моделями}. Следовательно, модель -- это объект, заместитель объекта-оригинала, обеспечивающий изучение некоторых свойств оригинала.

Замещение одного объекта другим с целью получения информации о важнейших свойствах объекта-оригинала с помощью объекта модели и называется \textbf{моделированием}.

В основе моделирования лежит \textbf{теория подобия}. Все модели можно разделить на три группы:

\begin{itemize}
    \item полное подобие (один в один);
    \item неполная (без учета некоторых факторов);
    \item приближенная.
\end{itemize}
