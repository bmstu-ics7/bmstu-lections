\section{Моделирования угроз и нарушителей}

\textbf{Уязвимость} -- это свойство системы, допускающее или способствующее реализации угрозы.

Цель моделирования угроз заставить разработчика конструктивно (на основе формального описания) мыслить при проектировании системы с точки зрения информационной безопасности.

\subsection{Стандарт моделирвоания угроз}

\begin{enumerate}
    \item[\textbf{1 этап.}] Определение активов (что?)
    \item[\textbf{2 этап.}] Описание архитектуры (где?)

        \begin{itemize}
            \item границы системы
            \item функциональность
            \item технологии
        \end{itemize}

    \item [\textbf{3 этап.}] Декомпозиция системы

        \begin{itemize}
            \item области защиты
            \item политики безопасности
        \end{itemize}

    \item[\textbf{4 этап.}] Определение угроз

        \begin{itemize}
            \item природные источники
            \item техногенные источники
            \item антропогенные источники
                \begin{itemize}
                    \item случайные
                    \item умышленные
                \end{itemize}
        \end{itemize}

    \item[\textbf{5 этап.}] Документирование угроз

        \begin{itemize}
            \item цель угрозы
            \item категория по STRIDE
                \begin{itemize}
                    \item spoofing (подлинность)
                    \item tampering (целостность)
                    \item repudation (потеря ответственности)
                    \item information disclousure (нарушение конфиденциальности)
                    \item denial of service (отказ доступа)
                    \item elevation of privilege (поднятие полномочий)
                \end{itemize}
        \end{itemize}

    \item[\textbf{6 этап.}] оценка / метод защиты

        \begin{itemize}
            \item DREAD

                \begin{itemize}
                    \item damage potential (что сломается)
                    \item reproducibility (воспроизводимость)
                    \item exploitability (используемость)
                    \item affected users (пострадавшие пользователи)
                    \item discoverability (возможность обнаружения)
                \end{itemize}
        \end{itemize}


\end{enumerate}

\subsection{Модель нарушителя}

\begin{itemize}
    \item низкий (может запускать разрешенные ему средства)
    \item средний (могжет запускать собственные средства, которые он должен протащить)
    \item высокий (может управлять системой)
    \item абсолютный (создатель системы)
\end{itemize}

\begin{enumerate}
    \item Увлеченные (мотивы: развлечение, слава, недооцененность на работе, доступ)
    \item Профессионалы (мотив: деньги)
\end{enumerate}

\subsection{Модели доступа}

\begin{itemize}
    \item HRU (дискретный доступ)
        \begin{table}[H]
            \centering
            \caption{Матрица доступа. Объект-субъект.}
            \label{tab:label}
            \begin{tabular}{|c|c|c|c|}
                \hline
                   & O1 & O2 & O3 \\
                \hline
                S1 & RW & & \\
                \hline
                S2 & & R & RWEX \\
                \hline
            \end{tabular}
        \end{table}

        \begin{table}[H]
            \centering
            \caption{Матрица доступа. Ролевая модель. Роли.}
            \label{tab:label}
            \begin{tabular}{|c|c|c|}
                \hline
                   & R1 & R2 \\
                \hline
                S1 & X & \\
                \hline
                S2 & X & X \\
                \hline
            \end{tabular}
        \end{table}

        \begin{table}[H]
            \centering
            \caption{Матрица доступа. Ролевая модель. Права ролей.}
            \label{tab:label}
            \begin{tabular}{|c|c|c|c|}
                \hline
                   & O1 & O2 & O3 \\
                \hline
                R1 & RW & & \\
                \hline
                R2 & & R & RWEX \\
                \hline
            \end{tabular}
        \end{table}

    \item Мандатная модель

        \begin{itemize}
            \item ССОВ -- совершенно секретно особой важности
            \item СС -- совершенно секретно
            \item С -- секретно
            \item ДСП -- для служебного пользования
        \end{itemize}

        \begin{table}[H]
            \centering
            \caption{Матрица доступа}
            \label{tab:label}
            \begin{tabular}{|c|c|c|c|c|}
                \hline
                & ССОВ & СС & С & ДСП \\
                \hline
                ССОВ & $\blacksquare$ & $\blacksquare$ & $\blacksquare$ & $\blacksquare$ \\
                \hline
                СС & & $\blacksquare$ & $\blacksquare$ & $\blacksquare$ \\
                \hline
                С & & & $\blacksquare$ & $\blacksquare$ \\
                \hline
                ДСП & & & & $\blacksquare$ \\
                \hline
            \end{tabular}
        \end{table}
\end{itemize}
