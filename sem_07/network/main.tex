\include{preamble/preamble}

\begin{document}

\include{titlepage/title}

\section{Курсовой проект}

\subsection{Направления}

\subsubsection{Желательно}

\begin{itemize}
    \item Разработка сетевого протокола для обмена данными между различными устройствами для решения сетевой задачи
    \item Система онлайн-трансляций (аудио- видео- конференцсвязь)
    \item Разработка драйвера для сетевой ОС для шифрования набора данных
    \item Система мониторинга сетевых сертификатов
    \item Приложение для удаленного управления устройством
    \item Система электронных транзакций (с поддержкой крипторграфических протоколов)
\end{itemize}

\subsubsection{Нежелательно}

\begin{itemize}
    \item Портал (сайт)
    \item Сетевое игровое приложение (часто реализуемые)
    \item Приложения для обмена файлами
    \item Чат
\end{itemize}

\subsection{Инструментарий}

\begin{enumerate}
    \item Среда разработки (для первых лр желательно C/C++)
    \item Виртуальный стенд
        \begin{itemize}
            \item Cisco Packet Tracer
            \item GNS
        \end{itemize}
    \item Wireshark
\end{enumerate}

\subsection{Литература}

\begin{itemize}
    \item Таненбаум <<Компьютерные сети>>
    \item Олифер, Олифер <<Компьютерные сети, принцип, технологии, протоколы>>
    \item Куроуз <<Компьютерные сети, нисходящие подходы>>
    \item Bees's Guide to Network Programming
    \item docs.microsoft.com
\end{itemize}

\subsection{Курсовая работа}

\begin{enumerate}
    \item Мат. модель
\end{enumerate}

\section{Классификация и обзор сетей}

\textbf{Компьютерная сеть} -- совокупность компьютеров и других устройств, соединенных линиями связи и обменивающихся информацией между собой в соответствии с определенными правилами -- протоколом.

\textbf{Протоколы} -- набор соглашений интерфейса логического уровня, которые определят обмен данными между разлиными программами. Эти соглашения задают единообразный способ передачи ошибок

\textbf{Сервер} -- специализированный компьютер и/или специализированное оборудование.



\end{document}
