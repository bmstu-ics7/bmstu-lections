\section{Введение}

Выделяют 3 модели нейронов:

\begin{enumerate}
    \item \textbf{Физиологические} -- нас не интересуют, понять как работает нейрон
    \item \textbf{Феноменологические} -- не рассмтариваем, они для биологов
    \item \textbf{Формальный нейрон} -- будем заниматься ими, математическая модель, попытка ее создать, не отражает работу физического нейрона.
\end{enumerate}

\subsection{Модель МакКалака Питтса}

\begin{equation*}
    y = \phi(\nu) = \frac{1}{1 + e^{-\alpha \nu}}
\end{equation*}

\begin{equation*}
    y = \th(\nu) = \frac{1 - e^{-2\alpha\nu}}{1 + e^{-2\alpha\nu}}
\end{equation*}

\begin{equation*}
    \frac{d \varphi}{d \nu} = \alpha \varphi(\nu)(1 - \varphi(\nu))
\end{equation*}

\begin{equation*}
    \frac{d \varphi}{d \nu} = \alpha (1 - \varphi^2(\nu))
\end{equation*}
