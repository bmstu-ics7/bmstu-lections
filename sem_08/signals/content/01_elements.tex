\section{Элементы теории сигналов}

\textbf{Сигнал} -- под сигналом понимается физический процесс отображающий сообщения и служащий для его передачи по каналу связи.

\subsection{Классификация}

\subsubsection{Критерии классификации}

\begin{itemize}
    \item множество значений, которые может принимать сигнал
    \item множество значений, которые принимают аргументы этого сигнала
\end{itemize}

В общем случае сигнал описывается функцией

\begin{equation*}
    U(x, y, z, t)
\end{equation*}

\begin{enumerate}
    \item Пространственный и временной
    \item Финитный и инфинитный
    \item Аналоговый и цифровой:

        \begin{itemize}
            \item Дискретный (аргументы не являются непрерывными, последовательность значений)
            \item Квантованный (аргументы конечные и дискретные)
        \end{itemize}

    \item Детерминированный и случайный
\end{enumerate}
