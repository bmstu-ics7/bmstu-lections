\section{Математичское представление сигналов}

\begin{equation*}
    U = U_{R_e} + i U_{im}, i \text{ -- мнимая единица}
\end{equation*}

\subsection{Формула Эйлера}

\begin{equation*}
    U = \underbrace{U_0}_{\text{амплитуда}} e^{i \underbrace{\varphi}_{\text{фаза}}}
\end{equation*}

\begin{equation*}
    |U|^2 \text{ -- интенсивность}
\end{equation*}

\subsection{Основные свойства}

\begin{enumerate}
    \item Степень отличия 2-х сигналов

        \begin{itemize}
            \item среднеквадратичное отклонение

                \begin{equation*}
                    d =  \sqrt{\sum_{i=0}^{N-1}  \mid U_1(x_i) - U_2(x_i) \mid^2}
                \end{equation*}

            \item максимальное отклонение

                \begin{equation*}
                    d = \max_{i=0,1, \ldots , N-1} \mid U_1(x_i) - U_2(x_i) \mid
                \end{equation*}

            \item PSNR -- пиковое отношение <<сигнал/шум>>

                Для изображений:

                \begin{equation*}
                    d = \lg \frac{255^2 N^2}{\sum_{i,j=0}^{N-1} \mid U_{1_{ij}} - U_{2_{ij}} \mid^2}
                \end{equation*}

            \item визуальный критерий
        \end{itemize}

    \item Принцип суперпозиции -- результат действия двух или более сигналов равен их геометрической сумме

        \begin{equation*}
            U = U_1 + U_2
        \end{equation*}

        \begin{equation*}
            U \ne \mid U_1 \mid^2 + \mid U_2 \mid^2
        \end{equation*}

    \item Разложение по базисным функциям

        \begin{equation*}
            U = \sum_{k=0}^{\infty} U_k \varphi_k
        \end{equation*}

        $\varphi_k$ -- базисные функции

        $U_k$ -- коэффициент разложения
\end{enumerate}

\subsection{Дискретиация сигналов}

\textbf{Дискретизация сигналов} -- это замена непрерывного сигнала последовательностью чисел, называемых отсчетами, являющийся представлением этого сигнала по некоторому базису.

\subsubsection{Теорема Котельникова}

Сигналы, сперкт Фурье которых равен нулю за пределами интервала $(-F; F)$, могут быть точно восстановлены по своим отсчетам взятым с шагом $\Delta t = \frac{1}{2F}$ по следующей формуле

\begin{equation*}
    U(t) = \sum_{k=-\infty}^{+\infty} U(k \Delta t) \text{sinc} \bigg( 2 \pi F \big(t - \frac{k}{2F}\big) \bigg)
\end{equation*}

$\text{sinc} (x) = \frac{\sin(x)}{x}$ -- функция отсчета

\paragraph{Спектр Фурье}

\begin{equation*}
    U(t) : V(f) = \int_{-\infty}^{+\infty} u(t) \exp (-2\pi ift) dt \text{ -- преобразования Фурье}
\end{equation*}

$f$ -- частота

\begin{enumerate}
    \item Муар-эффект $[-F, F]$
    \item Строб-эффект (Фильтрация с использованием <<окон>>)
\end{enumerate}

\subsubsection{Примеры}

\begin{enumerate}
    \item Окно Хэннинга

        \begin{equation*}
            w(f) =
            \begin{cases}
                \frac{1}{2} (1 + \cos(\pi \frac{f}{F})), \text{ при } \mid f \mid \le F \\
                0 \text{ иначе} \\
            \end{cases}
        \end{equation*}

    \item Окно Кайзера

        \begin{equation*}
            w(f) =
            \begin{cases}
                \frac{I_0 (\alpha \sqrt{1 - (\frac{f}{F})^2})}{I_0(\alpha)}, \mid f \mid \le F \\
                0 \text{ иначе} \\
            \end{cases}
        \end{equation*}

        $I_0$ -- модифицируемая функция Бесселя
\end{enumerate}

\subsection{Квантование}

\begin{equation*}
    \varepsilon_1 = \mid u - \tilde u_1 \mid
\end{equation*}

Функция потерь $D(\varepsilon_1)$

\begin{equation*}
    Q = \sum_{i=1}^{n} \int_{u_{i-1}}^{u_i} p(u)D(\varepsilon) du \rightarrow \min
\end{equation*}

\textbf{Предискажение сигнала} -- прежде чем выполнять операцию квантования мы подвергаем сигнал некоторому преобразованию.

\begin{equation*}
    \tilde u = w(u)
\end{equation*}

\begin{equation*}
    D(\Delta_i); \Delta_i = u_i - u_{i-1}
\end{equation*}

\begin{equation*}
    D(\Delta_i) =
    \begin{cases}
        1, \mid \Delta_i \mid \le \Delta_{\text{пор}} \\
        0, \mid \Delta_i \mid > \Delta_{\text{пор}} \\
    \end{cases}
\end{equation*}

\subsubsection{Закон Вебера-Фенера (для изображений)}

\begin{equation*}
    \Delta_{\text{пор}} = \delta_0 \cdot u
\end{equation*}

\begin{equation*}
    \frac{w(u) - w(u_{\min})}{w(u_{\max}) - w(u_{\min})} = \frac{\ln\big(\frac{u}{u_{\min}}\big)}{\ln q},\ q = \frac{u_{\max}}{u_{\min}}
\end{equation*}

\begin{equation*}
    N = \frac{\ln q}{\delta_0} \approx 230-240
\end{equation*}
