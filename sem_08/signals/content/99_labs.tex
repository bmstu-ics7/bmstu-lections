\section{Лабораторные работы}

\subsection{Лабораторная работа 1}

Дискретизация типовых сигналов

\begin{enumerate}
    \item Прямоугольный импульс

        \begin{equation*}
            u(x) = \text{rect}\big(\frac{x}{L}\big)
        \end{equation*}

        \begin{equation*}
            \text{rect}(x) =
            \begin{cases}
                1, \mid x \mid \le 1 \\
                0 \text{ иначе} \\
            \end{cases}
        \end{equation*}

    \item Функция Гаусса

        \begin{equation*}
            u(x) = A e^{-\frac{x^2}{\sigma^2}}
        \end{equation*}
\end{enumerate}

Два типовых сигнала: нужно сделать дискретизацию (получить набора значений в точках), а затем восстановить этот сигна по теореме Котельникова. Построить два графика: исходный сигнал, что получится в результате восстановления.
