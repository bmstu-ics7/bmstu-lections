\section{Лабораторные работы}

\subsection{Лабораторная работа 1}

Одноканальная система обслуживания.

Генератор $\rightarrow$ Буфер $\rightarrow$ Обслуживающий аппарат $\rightarrow$

Задан закон распределения поступления заявок ($\lambda$). Задан закон распределения времени обслуживания заявок ($\overline{t}_\text{обсл}$). Интервал прихода заявок.

\begin{equation*}
    t_\text{прихода} = \frac{1}{\lambda}
\end{equation*}

Пользователя интересует

\begin{equation*}
    \overline{t}_\text{пребывания} = \overline{t}_\text{ожидания} + \overline{t}_\text{обслуживания}
\end{equation*}

$\rho$ -- загрузка $\rho = \frac{\lambda}{\mu}$, $\lambda$ -- интенсивность поступления заявок, $\mu$ -- интесивность обслуживания заявок.

\begin{equation*}
    F = 1 - e^{-\lambda t}
\end{equation*}

\begin{equation*}
    F = 1 - e^{-\mu t}
\end{equation*}

\begin{enumerate}
    \item Пересчитать параметры заданого закона распределения таким образом, чтобы пользователь работал с интенсивностями.
    \item Построить график зависимости $\rho$ от среднего времени пребывания или ожидания.
\end{enumerate}

\begin{equation*}
    \overline{t}_\text{ож} = \frac{\rho}{(1 - \rho) \lambda}
\end{equation*}
