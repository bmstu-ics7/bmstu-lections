\chapter{Предельные теоремы теории вероятности}

\section{Неравенства Чебышева}

теорема: 1-е неравенство Чебышева

Пусть

\begin{enumerate}
    \item $X$ - слчайная величина
    \item $X \geq 0$ (т.е. $P\{X<0\} = 0$)
    \item $\exists MX$
\end{enumerate}

Тогда

$$
\forall \varepsilon > 0 P\{X \geq \varepsilon\} \le \frac{MX}{\varepsilon}
$$

Доказательство:

\begin{equation*}
    MX = \int_{-\infty}^{+\infty} xf_X(x) dx = \int_0^{+\infty} xf_X(x) dx = \text{свойство аддитивности} =
\end{equation*}

\begin{equation*}
    = \int_{0}^\varepsilon xf_X(x) dx + \int_\varepsilon^{+\infty} xf_X(x) dx \geq \int_\varepsilon^{+\infty} xd_X(x) dx \geq \varepsilon \int_\varepsilon^{+\infty} f_X(x) dx =
\end{equation*}

\begin{equation*}
    = \varepsilon P\{ X \geq \varepsilon \} \Rightarrow P\{ X \geq \varepsilon \} \le \frac{MX}{\varepsilon}
\end{equation*}

Теорема: 2-е неравенство Чебышева

Пусть

\begin{enumerate}
    \item $X$ - случайная величина
    \item $\exists MX, \exists DX$
\end{enumerate}

Тогда

\begin{equation*}
    \forall \varepsilon > 0 P\{ |X - MX | \geq \varepsilon\} \le \frac{DX}{\varepsilon^2}
\end{equation*}

Доказательство

\begin{equation*}
    DX = M[(X-MX)^2] =
    \left|
    \begin{matrix}
        1) Y = (X-MX)^2 \geq 0 \\
        2) \exists DX \Rightarrow \exists MX \Rightarrow \delta = \varepsilon^2 > 0 \\
        3) M[Y] \geq \delta P\{X \geq 0\} \\
    \end{matrix}
    \right| \geq \varepsilon^2 \cdot P\{Y \geq \varepsilon^2 \} =
\end{equation*}

\begin{equation*}
    = \varepsilon^2 P\{ (X - MX)^2 \geq \varepsilon^2 \} = \varepsilon^2 P\{|X-MX| \geq \varepsilon\} \Rightarrow
\end{equation*}

\begin{equation*}
    \Rightarrow P\{|X-MX| \geq \varepsilon \} \le \frac{DX}{\varepsilon^2}
\end{equation*}

\section{Виды сходимости последовательности случайных величин}

Пусть $X_1,X_2,...$ - последовательность случайных величин, заданных на олном вероятностном пространстве

Определение: Говорят, что последовательность $X-1, X_2,...$ сходится по вероятности к случайной величине $Z$, заданной на том же вероятностном пространстве, если

\begin{equation*}
    \forall \varepsilon > 0 P\{ |X_n - Z| \geq \varepsilon \} \to_{n \to \infty} 0
\end{equation*}

Обозначается

\begin{equation*}
    X_n \xrightarrow[n\to\infty]{P} Z
\end{equation*}

Замечание: Сходимость последовательности $X_n, n \in \mathcal N$, к случайной величине $Z$ по вероятности означает, что при достаточно больших $n$ вероятность отклонения случайной величины $X_n$ от случайной величины $Z$ будет меньше (любого) наперед заданного числа

Определение: Говорят, что последовательность $X_1, X_2,...$ слабо сходится к случайной величине $Z$, если в каждой точке $x$ непрерывности функции распределения $F_Z(x)$ числовая последовательность значений $F_{X_n}(x)$ сходится к $F_Z(x)$, то есть

\begin{equation*}
(\forall x \in \mathcal R) (( F_Z \text{непрерывная в точке } x ) \Rightarrow (F_{X_n} (x) \xrightarrow[n\to\infty]{} F_Z(x)))
\end{equation*}

Обозначается

\begin{equation*}
    X_n \Rightarrow_{n\to\infty Z}
\end{equation*}

\section{Закон больших чисел}

Определение: Говорят, что последовательность случайных величин $X_1,X_2,...$ удовлетворяет закону больших чисел, если

\begin{equation*}
    \forall \varepsilon > 0 P\{ \bigg|\frac{1}{n} \sum_{i=1}^n X_i - \frac{1}{n} \sum_{i=1}^n m_i\bigg| \geq \varepsilon \} \xrightarrow[n\to\infty]{} 0
\end{equation*}

Замечание:

\begin{enumerate}
    \item Рассмотрим случайную величину $\overline{X_n} = \frac{1}{n} \sum_{i=1}^n X_i$

        Выполнение для последовательности $X_1, X_2,...$ ЗБЧ означает, что при достаточно больших $n$ случайная величина $\overline{X_n}$ практически теряет случайный хаарктер и принимает неслучайное значение $\frac{1}{n} \sum_{i=1}^n m_i$ с вероятностью, близкой к 1

    \item Выполнение для последовательности $X_1,X_2,...$ ЗБЧ означает, что

        \begin{equation*}
            Y_n = \frac{1}{n} \sum_{i=1}^n (X_i - m_i) \to_{n\to\infty}^P 0
        \end{equation*}
\end{enumerate}

Вопрос: Какими свойствами должна обладать последовательность $X_1,X_2,...$, чтобы она удовлетворяла ЗБЧ?

Теорема: Чебышева (ЗБЧ в форме Чебышева)

Пусть

\begin{enumerate}
    \item $X_1,X_2,...$ -- последовательность независимых случайных величин
    \item $\exists MX_i = m_i, \exists DX_i = \sigma_i^2, i \in \mathcal N$
    \item Дисперсии ограничены в сов-ти, то есть

        \begin{equation*}
            \exists c > 0 \sigma_i^2 \le c, i \in \mathcal N
        \end{equation*}
\end{enumerate}

Тогда последовательность $X_1, X_2,...$ удовлетворяет ЗБЧ


Доказательство:

\begin{enumerate}
    \item Рассмотрим случайную величину

        \begin{equation*}
            \overline{X_n} = \frac{1}{n} \sum_{i=1}^n X_i, n \in \mathcal N
        \end{equation*}

        Тогда

        \begin{equation*}
            M[\overline{X_n}] = M[\frac{1}{n} \sum_{i=1}^n X_i] = \frac{1}{n} \sum_{i=1}^n MX_i = \frac{1}{n} \sum_{i=1}^n m_i
        \end{equation*}

        \begin{equation*}
            D[\overline{X_n}] = D[\frac{1}{n} \sum_{i=1}^n X_i] = \frac{1}{n^2} D[\sum_{i=1}^n X_i] = \frac{1}{n^2} \sum_{i=1}^n DX_i = \frac{1}{n^2} \sum_{i=1}^n \sigma_i^2
        \end{equation*}

    \item Запишем для случайной величины $\overline{X_n}$ 2-е неравенство Чебышева

        \begin{equation*}
            \forall \varepsilon > 0 P\{ |\overline{X_n} - M\overline{X_n} | \geq \varepsilon \} \le \frac{D\overline{X_n}}{\varepsilon^2}
        \end{equation*}

        То есть

        \begin{equation*}
            \forall \varepsilon > 0 P\{ | \frac{1}{n} \sum_{i=1}^n X_i - \frac{1}{n} \sum_{i=1}^n m_i | \geq \varepsilon \} \le \frac{1}{\varepsilon^2}\frac{1}{n^2} \sum_{i=1}^n \sigma^2_i 
        \end{equation*}

        Так как по условию дисперсии ограничены в сов-ти, то

        \begin{equation*}
            \sum_{i=1}^n \sigma_i^2 \le \sum_{i=1}^n C \le C \cdot n
        \end{equation*}

        поэтому

        \begin{equation*}
            \forall \varepsilon > 0 0 \le P \{ | \frac{1}{n} \sum_{i=1}^n X_i - \frac{1}{n} \sum m_i | \geq \varepsilon \} \le \frac{1}{\varepsilon^2} \frac{C \cdot n}{n^2}
        \end{equation*}

        Устремим $n \to \infty$

        По теореме о 2х милиционерах

        \begin{equation*}
            \forall \varepsilon > 0 P\{ |\frac{1}{n} \sum X_i - \frac{1}{n} \sum m_i| \geq \varepsilon \}  \xrightarrow[n\to\infty]{} 0
        \end{equation*}
\end{enumerate}

Замечание: Теорема Чебышева носит достаточный характер, то есть если для некоторой последовательности случайных величин выполнены ее условия, то последовательность удовлетворяет ЗБЧ. В этом случае также говорят, что последовательность удовлетворяет ЗБЧ в форме Чебышева. Если для последовательности не выполнены есловия теоремы Чебышева, то она не удовлетворяет ЗБЧ в форме Чебышева, но, возможно, удовлетворяет ЗБЧ.

Следствие:

Пусть

\begin{enumerate}
    \item $X_1, X_2, ...$ -- последовательность независимых одинаково распределенных случайных величин
    \item $m = MX_i, i \in \mathcal N$
\end{enumerate}

Тогда

\begin{equation*}
    \forall \varepsilon > 0 P\{ |\frac{1}{n} \sum X_i - m| \geq \varepsilon \}  \xrightarrow[n\to\infty]{} 0
\end{equation*}

Доказательство: Заметим, что $m_i \equiv m, i \in \mathcal N$, поэтому $\frac{1}{n} \sum m_i = m$, и исп. $m$ теоремы Чебышева

Следствие: ЗБЧ в форме Бернулли

Пусть

\begin{enumerate}
    \item Проводится $n$ испытаний по схеме Бернулли с вероятностью успеха $p$
    \item Наблюденная частота успеха
        \begin{equation*}
            r_n = \frac{\{\text{число успехов в серии n испытаний}\}}{n}, n \in \mathcal N
        \end{equation*}
\end{enumerate}

Тогда

\begin{equation*}
    r_n \xrightarrow[n\to\infty]{P} p
\end{equation*}

Доказательство

\begin{enumerate}
    \item Пусть

        \begin{equation*}
            X_i =
            \begin{cases}
                1, \text{ если в i-m испытании серии произошел успех} \\
                0, \text{ иначе}
            \end{cases}
        \end{equation*}

        \begin{table}[H]
            \centering
            \begin{tabular}{|c||c|c|}
                \hline
                $X_i$ & 0 & 1 \\
                \hline
                $P$ & $q$ & $p$ \\
                \hline
            \end{tabular}
        \end{table}

        где $q=1-p$ -- вероятность неудачи

        Тогда $\sum_{i=1}^n X_i$ -- число успехов в серии из $n$ испытаний

        \begin{equation*}
            \frac{1}{n} \sum_{i=1}^n X_i = r_n
        \end{equation*}

    \item 
    Применим к последовательности $X_1, X_2, ...$ ЗБЧ в форме Чебышева
    \begin{enumerate}
    \item $X_1, X_2, ...$ - независимы (определение схемы Бернулли)
    \item $MX_i=p,~DX_i=q,~i\in N$
    \item Так как $DX_i=pq$, то дисперсии определены в совокупности
    \end{enumerate}
    Применяя к последовательности следствие 1 получаем:
    
    \begin{equation*}
    \forall \varepsilon > 0 ~~ P\{\bigg |\underbrace{\frac{1}{n} \sum X_i}_{r_n} - \underbrace{m}_{p}\bigg| \geq \varepsilon \} \xrightarrow[n\to\infty]{} 0
    \end{equation*}
    
 Таким образом 
 \begin{equation*}
    r_n \xrightarrow[n\to\infty]{P} p
\end{equation*}

\end{enumerate}

\section{Центральная предельная теорема}

Рассмотрим последовательность $X_1, X_2, ...$ случайных величин, обладающих следующими свойствами:

\begin{enumerate}
    \item $X_i, i \in \mathcal N$ -- независимы
    \item $X_i, i \in \mathcal N$ -- одинаково распределены
    \item $\exists MX_i = m, \exists DX = \sigma_i^2, i \in \mathcal N$
\end{enumerate}

Обозначим:

\begin{equation*}
    \overline{X_n} = \frac{1}{n} \sum_{i=1}^n X_i
\end{equation*}

Тогда (смотреть расс-я из доказательсва теоремы Чебышева)

\begin{equation*}
    M\overline{X_n} = m
\end{equation*}

\begin{equation*}
    D\overline{X_m} = \frac{1}{n^2} n \cdot \sigma^2 = \frac{\sigma^2}{n}
\end{equation*}

Рассмотрим случайную величину

\begin{equation*}
    Y_n = \frac{\overline{X_n} - M\overline{X_n}}{\sqrt{D|\overline{X_n}}} = \frac{\overline{X_n} - m}{\sigma / \sqrt{n}}, n \in \mathcal N
\end{equation*}

\begin{equation*}
    MY_n = 0, DY_n = 1
\end{equation*}

Теорема ЦПТ

Пусть выполнены условия (1) -- (3). Тогда последовательность $Y_n$ слабо сходится к случайной величине $Z \sim N(0,1)$

То есть

\begin{equation*}
    (\forall x \in \mathcal R))(F_{Y_n}(x) \to_{n\to\infty} \Phi(x))
\end{equation*}

где

\begin{equation*}
    \Phi(x) = \frac{1}{\sqrt{2\pi}} \int_{-\infty}^{x} e^{-\frac{t^2}{2}} dt - \text{ функция распределения случайной величины } Z
\end{equation*}

Доказательство: без доказательства

Замечание:

\begin{enumerate}
    \item Следствие 1 ЗБЧ утверждает, что если выполнены (1) -- (3), то последовательность $\overline{X_n} \to_{n\to\infty}^P m$ ЦПТ уточняет характер этой сходимости

        Пример: ЭВМ производит суммирование $n=10^4$ чисел, каждое из которых округлено с точностью до $10^{-4}$. Считая, что ошибки окруления независимы и равномерно распределены на $(-0.5 \cdot 10^{-4}, 0.5 \cdot 10^{-4})$. Найти диапазон, в котором с вероятностью $0.95$ будет заключена ошибка всех суммы.

        \begin{enumerate}
            \item Пусть $X_i, i = \overline{1;n}$ - ошибка округления $i$-го числа

                \begin{enumerate}
                    \item все $X_i$ независимы (по условию)
                    \item $X_i \sim R(-0.5 \cdot 10^{-4}, 0.5 \cdot 10^{-4}) \Rightarrow$ одинаково распределены
                    \item
                        \begin{equation*}
                            MX_i = \frac{a+b}{2} = 0
                        \end{equation*}

                        \begin{equation*}
                            FX_i = \frac{(b-a)^2}{12} = \frac{(10^{-4})^{2}}{12} \approx 10^{-9}
                        \end{equation*}
                \end{enumerate}

                Таким образом последовательность $X_i$ удовлетворяет ЦПТ

            \item $S = \sum_{i=1}^m X_i$ -- ошибка всей суммы

                \begin{equation*}
                    Y_n = \frac{\frac{1}{n} \sum_{i=1}^n X_i - MX_i}{\sqrt{DX_i}/ n } \sim N(0,1)
                \end{equation*}

            \item Изображение с фото

                Таким образом $\varepsilon = U_{0.975}$ -- квантиль уровня $0.975$ распределения $N(0,1)$ из таблицы $U_{0.975} = 1.96$

                Таким образом

                \begin{equation*}
                    0.95 = P\{|Y_n| < 1.96 \} = P\{ |\frac{\frac{1}{n} S - 0}{\sqrt{DX_i / n}} | < 1.96 \} = P\{ |\frac{S}{\sqrt{DX_i}}| < 1.96 \} =
                \end{equation*}

                \begin{equation*}
                    = P \{ |S| < 1.96 \cdot \sqrt{nDX_i} \} = P\{|S| < 1.96 \sqrt{10^4 \cdot 10^{-9}}\} = P\{ |S| <  6.2 \cdot 10^{-3}\}
                \end{equation*}

                Ответ: с вероятностью 0.95
        \end{enumerate}
\end{enumerate}

Следствие. Интегральная Теорема Мдавра-Лапласа

Пусть:

\begin{enumerate}
    \item проводится $n>>1$ испытаний по схеме испытаний Бернулли с вероятностью успеха $p$
    \item $k$ -- число успехов в этой серии
\end{enumerate}

Тогда

\begin{equation*}
    P\{k_1 \le k \le k_2 \} \approx \Phi_0(x_2) - \Phi_0(x_1)
\end{equation*}

,где

\begin{equation*}
    x_i = \frac{k_i - np}{\sqrt{npq}}, i = \overline{1;2}
\end{equation*}

\begin{equation*}
    \Phi_0 (x) = \frac{1}{2\pi} \int_0^x e^{-\frac{t^2}{2}}dt, q = 1-p
\end{equation*}

Доказательство

\begin{enumerate}
    \item Рассмотрим случайную величину

        \begin{equation*}
            X_i =
            \begin{cases}
                1, \text{ если в i-м испытании произошел успех} \\
                0, \text{ иначе}
            \end{cases}
        \end{equation*}

        Тогда

        \begin{table}[H]
            \centering
            \begin{tabular}{|c||c|c|}
                \hline
                $X_i$ & 0 & 1 \\
                \hline
                $P$ & $q$ & $p$ \\
                \hline
            \end{tabular}
        \end{table}

        \begin{equation*}
            \left.
            \begin{matrix}
                MX_i = p, \\
                DX_i = pq \\
            \end{matrix}
            \right\}
            (*)
        \end{equation*}

    \item $k = \sum_{i=1}^n X_i$
        ФОТО
\end{enumerate}

\begin{example}
    В эксперименте Пирсона по подбрасыванию правильной монеты после $n=24000$ испытаний герб выпал $12012$ раз. Какова вероятность того, что при повторном испытании отклонение относительной частоты успеха будет таким же или больше

    Решение

    \begin{enumerate}
        \item Имеем схему испытаний Бернулли

            \begin{equation*}
                n=24000 >> 1
            \end{equation*}

            \begin{equation*}
                p = q = \frac{1}{2}
            \end{equation*}

            Пусть

            \begin{equation*}
                A = \{ \text{отклонение не меньше} \}
            \end{equation*}

            Тогда

            \begin{equation*}
                \overline A = \{ отклонение меньше \} = \{ |k-12000| < 12 \}
            \end{equation*}

            \begin{equation*}
                P(\overline A) = P \{ 119888 < k < 12012 \} = \Phi_0(x_2) - \Phi_0(x_1) \approx 2\Phi_0(0.155) \approx 0.123
            \end{equation*}

            \begin{equation*}
                x_2 = \frac{k_2 - np}{\sqrt{npq}} \approx 0.155
            \end{equation*}

            \begin{equation*}
                x_1 = \frac{k_1 - np}{\sqrt{npq}} = -0.155
            \end{equation*}

            \begin{equation*}
                P(A) = 1 - P(\overline A) \approx 0.877
            \end{equation*}
    \end{enumerate}
\end{example}
