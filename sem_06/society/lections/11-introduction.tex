\chapter{Общение как социально - социалогическая проблема}

\section{Вопросы}

\begin{enumerate}
    \item Понятие общение, структура и средство общения
    \item Характеристика и содержание общения и механизмы воздействия
    \item Поступки и самоподача общения
    \item Коммуникативная сторона общения
    \item Интерактивная сторона общения
    \item Стили общения
\end{enumerate}

\textbf{Общение} -- это сложный многоплановый процесс становления
и развития контактов между людьми, который пораждается потребностями
совсместной деятельности и включает в себя обмен информацией,
выработку совместной, единой стратегии взаимодействия, восприятия и 
понимаю другого человека

\section{3 стороны общения}

\begin{itemize}
    \item \textbf{Коммуникативная сторона} --
        обмен информацией между людей
    \item \textbf{Интерактивная сторона} --
        организация взаимодействия между индивидами
    \item \textbf{Перцептиваня сторона} --
        процесс восприятия друг друга
\end{itemize}

\section{2 вида общения}

\begin{itemize}
    \item \textbf{Вербальное} --
        общение при помощи слов (осознанное)
    \item \textbf{Невербальное} --
        общение при помощи жестов, поз, взглядов (неосознанное)

        \begin{itemize}
            \item \textbf{Кинетика} -- позы, жесты, мимика
                (эмоциональная реакция человека)

            \item \textbf{Паралингвистика} --
                особенности произношения, темб, высота
                голоса, громкость речи, смех, плач, ошибки

            \item \textbf{Проксемика} --
                это пространственное расположение
                собеседников относительно друг друга

            \item \textbf{Визуальный контакт} --
                контакт глаз
        \end{itemize}
\end{itemize}

\section{Основные механизмы познания человека}

\begin{itemize}
    \item \textbf{Идентификация} -- отождествление, уподобление
    \item \textbf{Эмпатия} --
        сопереживание, способность к постижению
        эмоционального состояния собеседника
    \item \textbf{Рефлексия} --
        обращение назад, знания того, как партнер понимаю меня
\end{itemize}

\section{Воздействие на партнера}

\begin{itemize}
    \item \textbf{Заражение} --
        бессознательное, невольное подверженность
        человека определенным психическим состоянием
    \item \textbf{Внушение} --
        целенаправленное неаргументированное
        воздействие одного человека на другого
        (авторитет -- главное при внушении)
    \item \textbf{Убеждение} --
        с помощью логического обоснования
        добиться согласия собеседника принять информацию
        (воздействие интеллектуальное)
    \item \textbf{Подражание} --
        принятие и воспроизведение внешних черт человека

        \begin{itemize}
            \item Подражание конкретному человеку
            \item Подражание нормам поведения группы
        \end{itemize}
\end{itemize}

\section{Схема восприятие}

\subsection{Фактор превосходства}

Если мы имеем дело с человеком, который нас превосходит, то мы его
переоцениваем.
Если мы имеем дело с человеком, которого мы чем-то превосходим,
то мы его недооцениваем

\subsection{Фактор привлекательности}

Чем более внешне привлекателен человек, тем более переоцениваем
его остальные качества и наоборот.

\subsection{Фактор отношения к нам}

Если человек, к нам хорошо относится, то мы его переоцениваем и
наоборот.

\subsection{Эффект ореола}

При формирование первого впечатления, общее позитивне впечатление
приводит к переоценке.

\section{Самоподача}

Управление вниманием может быть произвольным, а может
быть непроизвольным. \textbf{Самоподача} --
управление вниманием партнера.

Управление восприятием партнера происходит при
запуске следующих механизмов:

\begin{enumerate}
    \item \textbf{Самоподача превосходства}: признаки --
        знаки превосходства (одежда, манеры речи и поведения).
        Важна для профессионалов в общении с людьми
        (психологи, телеведущие и т.д.).

    \item \textbf{Самоподача привлекательности} --
        также является предметом управления. Самоподача
        привлекательности важна для всех. Правило простое:
        привлекательными нас делает та работа, которую мы
        провели с одеждой чтобы добиться привлекательности.

    \item \textbf{Самоподача отношения} -- есть вербальные
        и невербальные способы самоподачи.

    \item \textbf{Самоподача актуального состояния и причин поведения}
\end{enumerate}

Самоподача может послужить ошибок при восприятии другим
человеком. Коммуникация в общении важна для ее участников.

\section{Барьеры непонимания}

В процессе общения мы стремимся удовлетворить свои потребности.
Что препятствует эффективной коммуникации? Откуда появляются
барьеры и как их преодолеть?

\textbf{Барьеры непонимания} -- во многих ситуациях желания и
побуждения человека неправильно понимаются другими людьми.
Кажется, что человек «защищается» и возводит преграду от
наших переживаний. Любому человеку есть что защищать от
внешнего воздействия. Если поддаться влиянию человека, то
придется свое устройство мира менять (а это нужно что-то
делать и придется приспосабливаться к изменениям -- нет комфорта).

\subsection{Типы барьеров (механизмы защиты)}

\begin{enumerate}
    \item \textbf{Избегание} -- уклонение от контактов с
        нежелательными партнерами (защита от источников информации).
    \item \textbf{Авторитет} -- мы делим людей на 2 категории:
        авторитетные (доверяем им) и неавторитетные (не доверяем).
        Авторитетность зависит от социального положения, статуса,
        привлекательности, хорошего отношения и многого другого.
    \item \textbf{Непонимание} -- защита от самой информации.
\end{enumerate}

Систему барьеров можно охарактеризовать как автоматизированную
охранную систему. Но возможна «ложная тревога». Во многих
ситуациях эта система может навредить человеку (нужная и
актуальная информация не воспринимается). Для каждого человека
важно общаться так, чтобы его понимали окружающие, чтобы слова
не наталкивались на стену непонимания.

\subsection{Защита от барьеров}

\begin{enumerate}
    \item \textbf{От избегания} -- нужно чтобы было внимание со
        стороны слушателя, и чтобы оно было постоянным и не
        рассеивалось. Общение эффективно, если и говорящий и
        слушающий одинаково вкладываются в общение.

        \begin{enumerate}
            \item Прием привлечения внимания <<нейтральной фразы>> --
                фраза не связана с основной темы, но она имеет
                ценность для всех присутствующих.
            \item Второй прием -- <<прием завлечения>> -- говорящий
                произносит нечто трудно воспринимаемым образом.
                Слушающему приходится предпринимать специальные
                усилия, а эти усилия предполагают концентрацию внимания.
            \item Третий прием -- <<прием установления зрительного
                контакта>> -- очень часто используются во многих
                типах общения (и в массовых, и в личных). Прием
                обращения внимания также -- <<прием акцентировки>>
        \end{enumerate}
\end{enumerate}

\section{Стили общения}

Стиль общения определяется огромным списком факторов --
положением, мировоззрением, статуса и т.д.

\begin{enumerate}
    \item \textbf{Ритуальный}
    \item \textbf{Манипулятивный} -- когда к партнеру относятся
        как к средству достижения цели (им пользуются).
    \item \textbf{Гуманистический} -- доверительное,
        психотерапевтическое общение. Связано с
        настроенностью партнера. Это обоюдное внушение
        (оба партнера друг другу доверяют)
\end{enumerate}
